\section{Abstracting Programs with Parse Errors}
\label{sec:prog-abstract}

We start by introducing our approach for abstracting programs with parse errors
into a suitable sequence of tokens for training sequence classifiers. We explain
how a traditional Earley parser can be used to extract partial parses, along
with a Probabilistic Context-Free Grammar (PCFG), in order to get a better level
of abstraction with richer context information than the simple \emph{Lexer}
output.

\begin{figure}[t]
\small
\begin{minipage}[t]{0.42\linewidth}
  \lstDeleteShortInline{|}
  \[
  \boxed{
  \begin{array}{lcl}
    G              & \defeq  & (N,\ \Sigma,\ P,\ S) \\
    \pcfgsym       & \defeq  & (N,\ \Sigma,\ P,\ S,\ W) \\
    \midrule
    G'             & \defeq  & (N',\ \Sigma,\ P',\ S') \\
    P'             & \defeq  & P\ \cup\ \errorrulessym \\
    N'             & \defeq  & N\ \cup\ \{S',\ H,\ I\}\ \cup\\
                   &         & \{E_a\ |\ a \in \Sigma\} \\
    \midrule
    e              & \in    & L(G) \\
    e_{\bot}       & \notin & L(G) \\
    \pairsym       & \defeq & e_{\bot} \times e \\
    \datasetsym    & \defeq & \List{\pairsym} \\
  \end{array}
  }
  \]
  \lstMakeShortInline[language=Python, mathescape=true]{|}
\end{minipage}
\begin{minipage}[t]{0.57\linewidth}
  \lstDeleteShortInline{|}
  \[
  \boxed{
  \begin{array}{lcl}
    \ecepsym       & : & \errorrulessym \to e_{\bot} \to e \\
    \learnPCFGsym  & : & G \to \List{e} \to \pcfgsym \\
    \partialsym    & : & \pcfgsym \to e_{\bot} \to t^a  \\
    \midrule
    \trainDLsym    & : & \List{t^a \times \errorrulessym} \to \Model \\
    \predictDLsym  & : & \Model \to t^a \to \errorrulessym \\
    \midrule
    \diffsym       & : & \pairsym \to \errorrulessym \\
    \trainsym      & : & \datasetsym \to \Model \\
    \predictsym    & : & \Model \to G \to e_{\bot} \to \errorrulessym \\
    \midrule
    \systemsym     & : & G \to \datasetsym \to (e_{\bot} \to e) \\
  \end{array}
  }
  \]
  \lstMakeShortInline[language=Python, mathescape=true]{|}
\end{minipage}
\caption{A high-level API for building a \toolname system that learns parse
error repair function $e_{\bot} \to e$ from a $\datasetsym$ of program pairs.}
\label{fig:api}
\end{figure}


\mypara{Lexical Analysis}
\emph{Lexical analysis}, lexing or tokenization is the process of converting a
sequence of characters \ie a program into a sequence of tokens (strings with an
assigned and thus identified meaning). The program that performs lexical
analysis is called a \emph{lexer} and is usually combined with a parser, which
together analyze the syntax of a programming language $L$. However, when a
program has a syntax error, the output token sequence of the lexer is the only
level of abstraction that we can acquire for such a program, since the parser
returns an error.

\mypara{Token Sequences}
Our goal is to repair a \emph{program token sequence} $t_i$, which is a lexed
program with parse errors (\ie $t_i \notin L$), into a \emph{fixed token
sequence} $t_o \in L$ that can be used to return a repaired program without
syntax errors. Let $t_i$ be a sequence $T_1, T_2, \dots, T_n$ and $t_o$ be the
updated sequence $T_1, T_2, \dots, T_i, \dots, T_j, \dots, T_k$. The subsequence
$T_i, \dots, T_j$ is a part of $t_i$ that has been replaced, deleted or inserted
in order to generate the $t_o$. It can be the whole program, part of it or
multiple parts of it. The $t_o$ will finally be a token sequence that can be
parsed by the original language's $L$ parser.

However, programs can be large, \ie $n$ can take a really large value, which
makes it unsuitable for training effectively sequence models. These large
programs can also contain a lot of irrelevant information for fixing a specific
parse error, \eg if in our running example in \autoref{fig:orig-prog} there was
another correct function definition before the one with the parse error.
Therefore, our goal is to first generate a \emph{abstracted token sequence}
$t_a$ that removes all irrelevant information from $t_i$ and gives hints for the
parse error fix by using the internal states of an \emph{Earley} parser.


\subsection{Earley Partial Parses}
\label{sec:prog-abstract:partial}

We propose using an \emph{Earley parser} to generate the abstracted token
sequence $t_a$ for an input program sequence $t_i$. An Earley parser holds
internally a \emph{chart} data structure, \ie \emph{a list of partial parses}.
Given a production rule $X \rightarrow \alpha \beta$, the notation $X
\rightarrow \alpha \cdot \beta$ represents a condition in which $\alpha$ has
already been parsed and $\beta$ is expected and both are sequences of terminal
and non-terminal symbols.

Each state is a tuple $(X \rightarrow \alpha \cdot \beta,\ i)$, consisting of
\begin{itemize}
    \item the production currently being matched $(X \rightarrow \alpha \beta)$
    \item the current position in that production (represented by the dot
    $\cdot$)
    \item the position $i$ in the input at which the matching of this production
    began: the origin position
\end{itemize}

The state set at input position $k$ is called $S(k)$. The parser is seeded with
$S(0)$ consisting of only the top-level rule. The parser then repeatedly
executes three operations: \emph{prediction, scanning,} and \emph{completion}.
There exists a \emph{complete parse} if $(X \rightarrow \gamma \cdot, 0)$ ends
up in $S(n)$, where $(X \rightarrow \gamma)$ is the top level-rule and $n$ the
input length. We define as a \emph{partial parse} any partially completed rule,
\ie if there is $(X \rightarrow \alpha \cdot \beta, i)$ in some state $S(k)$,
where $i < k < n$.

Let, again, $T_1, T_2, \dots, T_i, \dots, T_k, \dots, T_n$ be the an input token
sequence $t_i$, where in location $k$ there is a parse error and the Earley
parser can not add any more rules in state $S(k + 1)$. We abstract $t_i$ by
getting the longest possible part of the program that has a partial parse, \ie
by finding the largest $i$ for which there is a rule $(X \rightarrow \alpha
\cdot \beta, 0) \in S(i)$. We use this rule for $X$ to replace $T_1, T_2, \dots,
T_i$ in $t_i$ with $\alpha$, thus getting an abstracted sequence $t_a$. We
continue abstracting the program until we reach the parse error and have no
states to use, \ie where $S(k + 1) = \emptyset$. In the same manner, we use the
longest possible partial parses that we can extract from the chart to abstract
$T_{i+1}, \dots, T_k$.

\mypara{Problem: Multiple Partial Parses} Each of the states $S(i)$, where $0
\leq i \leq k$, holds a large number of uncompleted rules, \ie partial parses.
We used as a heuristic to choose the longest possible partial parses to abstract
our erroneous programs and thus limiting the size of the program token sequence
$t_i$ as much as possible. However, it is possible that we have two or more such
parses in $S(k)$, where they have the same length, \eg $\{(X \rightarrow \alpha
\cdot \beta, i),\ (X' \rightarrow \alpha' \cdot \beta', i)\} \in S(k)$.


\mypara{Example}


\subsection{Probabilistic Context-Free Grammars}
\label{sec:prog-abstract:pcfg}

\mypara{Constructing a PCFG}

\mypara{Example of PCFG}

\mypara{Example}

\subsection{From Programs to Abstract Token Sequences}

\mypara{Explain Pseudocode combining PCFG + Most Likely}

\mypara{Example: Selecting Most Likely Partial Parse}



