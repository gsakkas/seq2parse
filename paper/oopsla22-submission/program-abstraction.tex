\section{Abstracting Programs with Parse Errors}
\label{sec:prog-abstract}

We start by introducing our approach for abstracting programs with parse errors
into a suitable sequence of tokens for training sequence classifiers. We explain
how a traditional Earley parser can be used to extract partial parses, along
with a Probabilistic Context-Free Grammar (PCFG), in order to get a better level
of abstraction with richer context information than the simple \emph{Lexer}
output.

\mypara{Lexical Analysis}
\emph{Lexical analysis}, lexing or tokenization is the process of converting a
sequence of characters \ie a program into a sequence of tokens (strings with an
assigned and thus identified meaning). The program that performs lexical
analysis is called a \emph{lexer} and is usually combined with a parser, which
together analyze the syntax of a programming language $L$. However, when a
program has a syntax error, the output token sequence of the lexer is the only
level of abstraction that we can acquire for such a program, since the parser
returns an error.

\mypara{Token Sequences}
Our goal is to repair a \emph{program token sequence} $t_i$, which is a lexed
program with parse errors (\ie $t_i \notin L$), into a \emph{fixed token
sequence} $t_o \in L$ that can be used to return a repaired program without
syntax errors. Let $t_i$ be a sequence $T_1, T_2, \dots, T_n$ and $t_o$ be the
updated sequence $T_1, T_2, \dots, T_i, \dots, T_j, \dots, T_k$. The subsequence
$T_i, \dots, T_j$ is a part of $t_i$ that has been replaced, deleted or inserted
in order to generate the $t_o$. It can be the whole program, part of it or
multiple parts of it. The $t_o$ will finally be a token sequence that can be
parsed by the original language's $L$ parser.

However, programs can be large, \ie $n$ can take a really large value, which
makes it unsuitable for training effectively sequence models. These large
programs can also contain a lot of irrelevant information for fixing a specific
parse error, \eg if in our running example in \autoref{fig:orig-prog} there was
another correct function definition before the one with the parse error.
Therefore, our goal is to generate first a \emph{abstracted token sequence}
$t_a$ that removes all irrelevant information from $t_i$ and gives hints for the
parse error fix by using the internal states of an \emph{Earley} parser.


\subsection{Earley Partial Parses} \label{sec:prog-abstract:partial}

\mypara{Problem: Multiple Partial Parses}

\mypara{Example}


\subsection{Probabilistic Context-Free Grammars}
\label{sec:prog-abstract:pcfg}

\mypara{Constructing a PCFG}

\mypara{Example of PCFG}

\mypara{Example}

\subsection{From Programs to Abstract Token Sequences}

\mypara{Explain Pseudocode combining PCFG + Most Likely}

\mypara{Example: Selecting Most Likely Partial Parse}



