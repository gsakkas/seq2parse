\section{Introduction}
\label{sec:intro}

% Context
Parsing is important and difficult.

% Gap
There are highly engineered approaches, \eg parser or pylint, that are accurate
but require a lot of manual effort from the software engineers.

Recent literature in the Natural Language Processing (NLP) research domain has
suggested many automatic approaches for natural languages, \eg language
translation. They have been applied on PL research on various applications, \eg
machine translation, but have been inaccurate by themselves.

Relevant work in PL research has suggested error correcting parsers (EC-Parsers)
for parsing programs with syntax errors with high accuracy and minimal effort
from the developer. The drawback of this approach, however, is that is extremely
slow and hasn't been improved for years.

% Innovation
\mypara{\toolname}
We propose training sequence classifiers for predicting error rules for
EC-Parsers in order to automatically, accurately and efficiently parse programs
with syntax errors.

% Results
We implemented our approach in \toolname and trained it and tested it on a
dataset of more than 1,100,000 programs. Given a new ill-typed program,
\toolname generates a list of potential solutions ranked by likelihood and an
\emph{edit-distance} metric. We train \toolname on programs from one year and
evaluate in several ways.
%
First, we measure its \emph{accuracy}: we show that \toolname correctly predicts
the right repair template 69\% of the time when considering the top three
templates and surpasses 80\% when we consider the top six.
%
Second, we measure its \emph{efficiency}: we show that \toolname is able to
synthesize a concrete repair within 20 seconds 70\% of the time.
%
Finally, we measure the \emph{quality} of the generated messages via a user
study with 29 participants and show that humans perceive both \toolname's edit
locations and final repair quality to be better than those produced by \seminal,
a state-of-the-art OCaml repair tool \citep{Lerner2007-dt} in a
statistically-significant manner.

% % The problem
% %
% Languages with Hindley-Milner style, unification-based inference offer the
% benefits of static typing with minimal annotation overhead. The catch, however,
% is that programmers must first ascend the steep learning curve associated with
% understanding the \emph{error messages} produced by the compiler.
% %
% While \emph{experts} can, usually, readily decipher the errors, and view them as
% invaluable aids to program development and refactoring, \emph{novices} are
% typically left quite befuddled and frustrated, without a clear idea of
% \emph{what} the problem is~\citep{Wand1986-nw}.
% %
% Owing to the importance of the problem, several authors have proposed methods to
% help debug type errors, typically, by \emph{slicing} down the program to the
% problematic locations~\citep{Haack2003-vc, Rahli2015-tt}, by \emph{enumerating}
% possible causes \citep{Lerner2007-dt, Chen2014-gd}, or by \emph{ranking} the
% possible locations using MAX-SAT \citep{Pavlinovic2014-mr},
% Bayesian~\citep{Zhang2014-lv} or statistical analysis~\citep{Seidel:2017}.
% %
% While valuable, these approaches at best help localize the problem but students
% are still left in the dark about how to \emph{fix} their code.

% % Repairs as error messages?
% \mypara{Repairs as Feedback}
% %
% Several recent papers have proposed an inspiring new line of
% attack on the feedback problem: using techniques from synthesis
% to provide feedback in the form of \emph{repairs} that students
% can apply to improve their code.
% %
% These repairs are found either by symbolically searching a space of candidate
% programs circumscribed by an expert-defined repair model
% \citep{singh2013,HeadGSSFDH17}, or via the observation that \emph{similar
% programs} have similar repairs, \ie by calculating ``diffs'' from the given
% solution and a \emph{correct} program that is the ``closest'' to the student's
% solution ~\citep{Gulwani_2018,Wang_2018}

% While this approach is compelling for generating feedback in large
% classes, it has several crucial requirements that render it inapplicable
% for type error messages.
% %
% First, for type errors, the space of candidate repairs is massive.
% It is quite unclear whether a small set of repair models \emph{exists}
% or even if it does, what it \emph{looks like}. More importantly,
% to scale, it is essential that we remove the requirement that an
% expert carefully curate some set of candidate repairs.
% %
% Second, the approach requires a corpus of similar programs,
% whose syntax trees or execution traces can be used to match
% each incorrect program with a ``correct'' version that is
% used to provide feedback. Programs with static type errors
% have no execution traces.
% %
% More importantly, we desire a means to generate feedback
% for \emph{new} programs that novices write, and hence
% cannot rely on matching against some (existing) correct
% program.

% \mypara{Analytic Program Repair}
% %
% In this work, we present a novel error repair strategy called \emph{Analytic
% Program Repair} that uses supervised learning instead of manually crafted repair
% models or matching against a corpus of correct code.
% %
% Our strategy is based on the key insight that \emph{similar errors} have similar
% repairs and realizes this insight by using a training dataset of pairs of
% ill-typed programs and their fixed versions to:
% %
% (1)~\emph{learn} a collection of candidate repair templates
%     by abstracting and partitioning the edits made in the
%     training set into a representative set of templates;
% %
% (2)~\emph{predict} the appropriate template from a given error,
%     by training multi-class classifiers on the repair templates
%     used in the training set;
% %
% (3)~\emph{synthesize} a concrete repair from the template
%    by enumerating and ranking correct (\eg well-typed)
%    terms matching the predicted template,
% %
% thereby, generating a fix for a candidate program.
% %
% Critically, we show how to perform the crucial abstraction
% from a particular \emph{program} to an abstract \emph{error}
% by representing programs via \emph{bag-of-abstracted-terms} (BOAT)
% \ie as numeric vectors of syntactic and semantic features \citep{Seidel2017-ko}.
% %
% This abstraction lets us train predictors over high-level
% code features, \ie to learn correlations between features
% that cause errors and their corresponding repairs, allowing
% the analytic approach to generalize beyond matching against
% existing programs.

% \mypara{\toolname}
% %
% We have implemented our approach in \toolname: a type error reporting
% tool for \ocaml programs. We train (and evaluate) \toolname on a set of
% over 4,500 ill-typed \ocaml programs drawn from two years of an
% introductory programming course.
% %
% Given a new ill-typed program, \toolname generates a list of potential
% solutions ranked by likelihood and an \emph{edit-distance} metric.
% We train \toolname on programs from one year and evaluate in several
% ways.
% %
% First, we measure its \emph{accuracy}: we show that \toolname correctly predicts
% the right repair template 69\% of the time when considering the top three
% templates and surpasses 80\% when we consider the top six.
% %
% Second, we measure its \emph{efficiency}: we show that \toolname is able to
% synthesize a concrete repair within 20 seconds 70\% of the time.
% %
% Finally, we measure the \emph{quality} of the generated messages via a user
% study with 29 participants and show that humans perceive both \toolname's edit
% locations and final repair quality to be better than those produced by \seminal,
% a state-of-the-art OCaml repair tool \citep{Lerner2007-dt} in a
% statistically-significant manner.
